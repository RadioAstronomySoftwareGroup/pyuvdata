\documentclass[11pt, oneside]{article}   	% use "amsart" instead of "article" for AMSLaTeX format
\usepackage{geometry}                		% See geometry.pdf to learn the layout options. There are lots.
\geometry{letterpaper}                   		% ... or a4paper or a5paper or ... 
%\geometry{landscape}                		% Activate for for rotated page geometry
%\usepackage[parfill]{parskip}    		% Activate to begin paragraphs with an empty line rather than an indent
\usepackage{graphicx}			
							
\usepackage{amssymb}

\usepackage{hyperref} 
\hypersetup{
    colorlinks = true
}

\title{Memo: UVBeam FITS Format}
\author{Bryna Hazelton, and the pyuvdata team}
\date{Jan 27, 2018}							% Activate to display a given date or no date

\begin{document}
\maketitle
\section{Introduction}
This memo introduces a new FITS file format for storing
beam models associated with the UVBeam object in
pyuvdata\footnote{\url{https://github.com/HERA-Team/pyuvdata}}, a python package that
provides an interface to interferometric data. Here, we describe the required and optional elements
and the structure of a UVBeam FITS (hereafter \textit{beamfits}) file. 

The contents of \textit{beamfits} files are explicitly mapped to attributes of the UVBeam object in 
pyuvdata. For more details on these parameters, please see 
\url{http://pyuvdata.readthedocs.io/en/latest/uvbeam_parameters.html}.
For examples of how to interact with this format using pyuvdata, please see the
pyuvdata tutorial: \url{http://pyuvdata.readthedocs.io/en/latest/tutorial.html}.

\section{Overview}
There are two main types of \textit{beamfits} files, depending on whether the beams are 
pixelized in HEALPix\footnote{\url{https://healpix.jpl.nasa.gov/}} or a regular pixel grid. There are also two subtypes of files for power 
and E-field beams for each pixelization scheme. So there are four primary flavors of 
\textit{beamfits} files which have small differences in format. There are also several optional 
metadata components that may or may not be present in any given file. These variations are 
described in detail in the following sections.

\section{Primary HDU}
The primary HDU of \textit{beamfits} files is an Image HDU containing the beam model.

\subsection{Primary Header}
The following are required keywords in the primary header of a  \textit{beamfits} file. For more detailed explanations 
of what these keywords mean, see the descriptions on pyuvdata's ReadTheDocs uvbeam\_parameters page. 
The UVBeam parameter corresponding to each keyword is noted in parentheses.
As with all FITS files, HISTORY and COMMENT cards are allowed.

\begin{itemize}
\item{\textbf{BTYPE}: should be set to either `power' or `efield' to designate a power or E-field beam type. (beam\_type)}
\item{\textbf{NORMSTD}: should be set to one of `physical', `peak' or `solid\_angle'. (data\_normalization)}
\item{\textbf{COORDSYS}: should be set to one of `az\_za', `sin\_zenith' or `healpix'. (pixel\_coordinate\_system)}
\item{\textbf{TELESCOP}: Telescope name. (telescope\_name)}
\item{\textbf{FEED}: Feed name (used to distinguish between different physical feed types on the same telescope). (feed\_name)}
\item{\textbf{FEEDVER}: Feed version (used to distinguish between different versions of the same feed on the same telescope). (feed\_version)}
\item{\textbf{MODEL}: Model name  (used to distinguish between different models of the same hardware). (model\_name)}
\item{\textbf{MODELVER}: Model version (used to distinguish between different versions of the same models of the same hardware). (model\_version)}
\item{\textbf{FEEDLIST}: Only for E-field beams. The list of feeds orientations represented in the data. (feed\_array)}
\item{\textbf{NSIDE}: Only for HEALPix beams. The NSIDE of the HEALPix map. (nside)}
\item{\textbf{ORDERING}: Only for HEALPix beams. The ordering parameter of the HEALPix map.(ordering)}
\end{itemize}

\subsection{Beam Data}
The axes of the data array in the primary HDU depends on the pixelization scheme.

\subsubsection{HEALPix beam axes}
For HEALPix beams, the data axes are: 
\begin{enumerate}
\item{\textbf{PIX\_IND}: index into the HEALPix pixel array (in the HPX\_INDS extension) This is needed because partial sky maps are supported so pixels may be missing.}
\item{\textbf{FREQ}: frequency}
\item{\textbf{Polarization axis (STOKES or FEEDIND)}: STOKES for power beams give the polarization integer convention from uvfits (see AIPS memo 117), FEEDIND for E-field beams give the index into the FEEDLIST.}
\item{\textbf{IF}: Spectral window number, currently only one spectral window is supported so this can only have length 1.}
\item{\textbf{VECIND}: Basis vector index, the index into the dimensions of the basis vector array (stored in the BASISVEC extension). This is usually length 1 for power beams and length 2 for E-field beams.}
\item{\textbf{COMPLEX}: real and imaginary (axis length 2)}
\end{enumerate}

\subsubsection{regularly pixelized beam axes}
For regularly pixelized beams, the data axes are: 
\begin{enumerate}
\item{\textbf{First Image Axis (AZ or SIN\_X)}: Coordinates for the first image axis. The axis name is dictated by the COORDSYS keyword.}
\item{\textbf{Second Image Axis (ZA or SIN\_Y)}: Coordinates for the second image axis. The axis name is dictated by the COORDSYS keyword.}
\item{\textbf{FREQ}: frequency}
\item{\textbf{Polarization axis (STOKES or FEEDIND)}: STOKES for power beams give the polarization integer convention from uvfits (see AIPS memo 117), FEEDIND for E-field beams give the index into the FEEDLIST.}
\item{\textbf{IF}: Spectral window number, currently only one spectral window is supported so this can only have length 1.}
\item{\textbf{VECIND}: Basis vector index, the index into the dimensions of the basis vector array (stored in the BASISVEC extension). This is usually length 1 for power beams and length 2 for E-field beams.}
\item{\textbf{COMPLEX}: real and imaginary (axis length 2)}
\end{enumerate}

\section{BASISVEC HDU}
The BASISVEC HDU is a required element of \textit{beamfits} files. It is an Image HDU containing the basis vector array.
The \textbf{EXTNAME} keyword in the header must be set to BASISVEC. The only other required keyword in the header is:

\begin{itemize}
\item{\textbf{COORDSYS}: should be set to one of `az\_za', `sin\_zenith' or `healpix'. (pixel\_coordinate\_system) and must match the corresponding keyword in the primary header (this is used as a consistency check)}
\end{itemize}

\subsection{Basis Vector Data}
The axes of the data array in the BASISVEC HDU depends on the pixelization scheme.

\subsubsection{HEALPix beam axes}
For HEALPix beams, the data axes are: 
\begin{enumerate}
\item{\textbf{PIX\_IND}: index into the HEALPix pixel array (in the HPX\_INDS extension) This is needed because partial sky maps are supported so pixels may be missing.}
\item{\textbf{AXISIND}: vector coordinate system axis index. For HEALPix beams, the two axes are azimuth and zenith angle in that order.}
\item{\textbf{VECCOORD}: Basis vector index, the index into the dimensions of the basis vector array (stored in the BASISVEC extension). This is usually length 1 for power beams and length 2 for E-field beams.}
\end{enumerate}

\subsubsection{regularly pixelized beam axes}
For regularly pixelized beams, the data axes are: 
\begin{enumerate}
\item{\textbf{First Image Axis (AZ or SIN\_X)}: Coordinates for the first image axis. The axis name is dictated by the COORDSYS keyword.}
\item{\textbf{Second Image Axis (ZA or SIN\_Y)}: Coordinates for the second image axis. The axis name is dictated by the COORDSYS keyword.}
\item{\textbf{AXISIND}: vector coordinate system axis index. For regularly pixelized beams, the coordinate system matches the pixel coordinate system ([az, za] or [sin\_x, sin\_y]).}
\item{\textbf{VECCOORD}: Basis vector index, the index into the dimensions of the basis vector array (stored in the BASISVEC extension). This is usually length 1 for power beams and length 2 for E-field beams.}
\end{enumerate}

\section{HPX\_INDS HDU}
The HPX\_INDS HDU is a required element of HEALPix \textit{beamfits} files. It is a Binary Table HDU containing the HEALPix pixel numbers.
The \textbf{EXTNAME} keyword in the header must be set to HPX\_INDS. There are no other required keywords in the header.

\subsection{HEALPix Pixel Table}
There is only one required column in the HPX\_INDS table:

\begin{itemize}
\item{\textbf{HPX\_INDS}: The HEALPix pixel numbers in the order that they are stored in the beam and basis vector arrays.}
\end{itemize}

\section{BANDPARM HDU}
The BANDPARM HDU is an required element of \textit{beamfits} files. It is a Binary Table HDU containing the bandpass and other optional parameters that are functions of frequency only.
The \textbf{EXTNAME} keyword in the header must be set to BANDPARM. There are no other required keywords in the header, but there are 2 optional header keywords:

\begin{itemize}
\item{\textbf{REFZIN}: Reference input impedance of the receiving chain (sets the reference for the S parameters), units: Ohms}
\item{\textbf{REFZOUT}: Reference output impedance of the receiving chain (sets the reference for the S parameters), units: Ohms}
\end{itemize}

\subsection{BANDPARM Table}
BANDPASS is the only required column in the BANDPARM table, but there are several optional columns that may be present.

\begin{itemize}
\item{\textbf{BANDPASS}: Frequency dependence of the beam. Depending on the data\_normalization, this may contain only the frequency dependence of the receiving chain or all the frequency dependence.}
\item{\textbf{RX\_TEMP}: Receiver temperatures.}
\item{\textbf{LOSS}: Antenna losses.}
\item{\textbf{MISMATCH}: Antenna-amplifier mismatch.}
\item{\textbf{S11}: S11 parameter of the receiving chain.}
\item{\textbf{S12}: S12 parameter of the receiving chain.}
\item{\textbf{S21}: S21 parameter of the receiving chain.}
\item{\textbf{S22}: S22 parameter of the receiving chain.}
\end{itemize}


\end{document}  
\documentclass[11pt, oneside]{article}   	% use "amsart" instead of "article" for AMSLaTeX format
\usepackage{geometry}                		% See geometry.pdf to learn the layout options. There are lots.
\geometry{letterpaper}                   		% ... or a4paper or a5paper or ... 
%\geometry{landscape}                		% Activate for for rotated page geometry
%\usepackage[parfill]{parskip}    		% Activate to begin paragraphs with an empty line rather than an indent
\usepackage{graphicx}			
							
\usepackage{amssymb}

\usepackage{hyperref} 
\hypersetup{
    colorlinks = true
}

\title{Memo: UVBeam FITS Format}
\author{Bryna Hazelton, and the pyuvdata team}
\date{Jan 27, 2018}							% Activate to display a given date or no date

\begin{document}
\maketitle
\section{Introduction}
This memo introduces a new FITS file format for storing
beam models associated with the UVBeam object in
pyuvdata\footnote{\url{https://github.com/HERA-Team/pyuvdata}}, a python package that
provides an interface to interferometric data. Here, we describe the required and optional elements
and the structure of a UVBeam FITS (hereafter \textit{beamfits}) file. 

The contents of \textit{beamfits} files are explicitly mapped to attributes of the UVBeam object in 
pyuvdata. For more details on these parameters, please see 
\url{http://pyuvdata.readthedocs.io/en/latest/uvbeam_parameters.html}.
For examples of how to interact with this format using pyuvdata, please see the
pyuvdata tutorial: \url{http://pyuvdata.readthedocs.io/en/latest/tutorial.html}.

\section{Overview}
There are two main types of \textit{beamfits} files, depending on whether the beams are 
pixelized in HEALPix\footnote{\url{https://healpix.jpl.nasa.gov/}} or a regular pixel grid. There are also two subtypes of files for power 
and E-field beams for each pixelization scheme. So there are four primary flavors of 
\textit{beamfits} files which have small differences in format. There are also several optional 
metadata components that may or may not be present in any given file. These variations are 
described in detail in the following sections.

\section{Primary HDU}
The primary HDU of \textit{beamfits} files is an Image HDU containing the beam model.

\subsection{Primary Header}
The following are required keywords in the primary header of a  \textit{beamfits} file. For more detailed explanations 
of what these keywords mean, see the descriptions on pyuvdata's ReadTheDocs uvbeam\_parameters page. 
The UVBeam parameter corresponding to each keyword is noted in parentheses.
As with all FITS files, HISTORY and COMMENT cards are allowed.

\begin{itemize}
\item{\textbf{BTYPE}: should be set to either `power' or `efield' to designate a power or E-field beam type. (beam\_type)}
\item{\textbf{NORMSTD}: should be set to one of `physical', `peak' or `solid\_angle'. (data\_normalization)}
\item{\textbf{COORDSYS}: should be set to one of `az\_za', `sin\_zenith' or `healpix'. (pixel\_coordinate\_system)}
\item{\textbf{TELESCOP}: Telescope name. (telescope\_name)}
\item{\textbf{FEED}: Feed name (used to distinguish between different physical feed types on the same telescope). (feed\_name)}
\item{\textbf{FEEDVER}: Feed version (used to distinguish between different versions of the same feed on the same telescope). (feed\_version)}
\item{\textbf{MODEL}: Model name  (used to distinguish between different models of the same hardware). (model\_name)}
\item{\textbf{MODELVER}: Model version (used to distinguish between different versions of the same models of the same hardware). (model\_version)}
\item{\textbf{FEEDLIST}: Only for E-field beams. The list of feeds orientations represented in the data. (feed\_array)}
\item{\textbf{NSIDE}: Only for HEALPix beams. The NSIDE of the HEALPix map. (nside)}
\item{\textbf{ORDERING}: Only for HEALPix beams. The ordering parameter of the HEALPix map.(ordering)}
\end{itemize}

\subsection{Beam Data}
The axes of the data array in the primary header depends on the pixelization scheme.

For HEALPix beams, the data axes are:


\end{document}  